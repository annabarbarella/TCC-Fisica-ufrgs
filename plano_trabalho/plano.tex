\documentclass{article}
\usepackage[english]{babel}
\usepackage[utf8]{inputenc}
\usepackage{graphicx}
\usepackage{color}
\usepackage{scalefnt}
\usepackage{hyperref}
\usepackage{float}
\usepackage[margin=1.2in]{geometry}
\hypersetup{
    colorlinks,
    citecolor=black,
    filecolor=black,
    linkcolor=black,
    urlcolor=black
}

\title{Projeto de Trabalho de Conclusão de Curso em Física Bacharel em Astrofísica}

\author{Universidade Federal do Rio Grande do Sul - UFRGS, Instituto de Física - IF}
\scalefont{1.0}
\begin{document}
\date{}
\maketitle
\begin{center}
\textbf{ Título do trabalho : Aplicação de métodos estatísticos para detecção e caracterização de populações estelares} \\
\textbf{Aluna:} Anna Bárbara de Andrade Queiroz \\
\textbf{Orientador:} Basílio Xavier Santiago \\
\textbf{Colaboradores:} Elmer Luque, Eduardo Balbinot, DES-Brasil
\end{center}

\section{Introdução}

Apresentamos neste documento a descrição do plano para o Trabalho de Conclusão de Curso (TCC) da aluna  Anna Bárbara Queiroz. A pesquisa a ser apresentada é centrada na detecção e também caracterização de substruturas localizadas no Halo Galáctico. \\
Essas subestruturas são grupos de estrelas que compartilham um vínculo de formação, como aglomerados globulares, galáxias anãs e correntes estelares, tais sistemas também entram dentro do conceito de população estelar. \\
 A detecção e caracterização dessas subestruturas propulsiona diversas outras pesquisas em astrofísica como uma melhor descrição e modelagem do componente do halo Galáctico, construção do histórico de acreção de massa da Galáxia, modelamento do potencial gravitacional da Via Láctea e também estudo de cenários de formação estelar. \\
Para a detecção dessas estruturas utilizamos a técnica chamada de Matched Filter que consiste de uma comparação estatística (variância) entre um modelo de população estelar simples (SSP) simulada e uma amostra de estrelas. A caracterização é feita através do ajuste de isócronas (modelos de evolução estelar),  e ajuste de perfis de densidade. \\
Este trabalho será baseado nos dados do projeto Dark Energy Survey
(DES, http://www.darkenergysurvey.org/) e do Sloan Digital Sky Survey III (SDSS-III, York at al, 2000;APJ;120-1587), projetos com o qual a aluna em questão esteve bastante envolvida durante a sua bolsa de iniciação científica. \\
Também aplicaremos um método para o cálculo de distâncias de estrelas. Usando uma estatística Bayesiana, porém agora combinando espectrometria e fotometria. As distâncias das estrelas são fundamentais no estudo de populações estelares e sua variação através da Galáxia. \\
Descrevemos na seção objetivos, os principais pontos a serem alcançados com a pesquisa, na seção metodologia sintetizamos o modo de execução do trabalho e as ferramentas que serão utilizadas e na seção cronograma organizamos as principais tarefas cronologicamente.

\section{Objetivos}

Os objetivos principais do trabalho são:

1) O manuseio de ferramentas estatísticas para detecção de populações estelares que se encontram no halo da Galáxia.

2) Validação do método, detectando subestruturas já conhecidas, com diferentes parâmetros.

3) Aplicação do método nos dados do DES e do SDSS-III, onde esperamos encontrar novas subestruturas.

4) Caracterização dos objetos, ajuste de diagramas cor-magnitude (CMD), suavização de mapas de densidade e ajuste de perfis.

5) Uso de métodos estatísticos Bayesianos, combinando espectrometria e fotometria para relacionar parâmetros de modelos com observações.


\section{Metodologia}

 Para a detecção das populações usaremos o código Sparse (1) que calculando a variância entre um modelo e uma amostra de estrelas, nos dá a probabilidade de cada estrela dentro dessa amostra pertencer ao modelo. Com esta probabilidade construiremos um mapa de densidade a ser analisado. Para a análise dos mapas usaremos o detector de sobredensidades SExtractor(6). \\
Serão simulados diversos modelos para uma busca sistemática, estas simulações são feitas por um gerador randômico de estrelas que utiliza modelos de evolução estelar  de Padova(2) e uma função de massa inicial (3). \\
Para a parte de caracterização, os candidatos a população serão analisados visualmente, ajustaremos isócronas para associar parâmetros como idade metalicidade e distância. Além disso analisaremos os mapas de densidade também visualmente e quando necessário faremos um ajuste do  perfil de densidade. \\
Para relacionar parâmetros como distâncias de estrelas, usando espectrometria e fotometria, seguiremos a ideia descrita em Santiago et al (2015) (5).

\section{Cronograma}

Manho - Junho (2015) : Simulação de modelos e adaptação com o código Sparse, fazer melhorias e scripts para uma busca sistemática.\\
Junho - Julho (2015) : Validar o método em populações estelares já conhecidas, usando dados do SDSS-III. \\
Julho - Agosto (2015) : Executar o código para amostras do DES, dados novos a serem liberados e também antigos. Executar também para amostras estelares do SDSS-III onde podem ser detectados novos objetos.\\
Agosto - Outubro (2015) : Analisar as subestruturas desconhecidas, caracterizá-las usando diagramas cor magnitude \\
Outubro - Novembro (2015) : Preparar código que usa estatística Bayesiana combinando dados espectro-fotométricos para estimar parâmetros como distância e avermelhamento para estrelas. Análise de resultados

\section{Referências Bibliográficas}
(1) Balbinot et al (2011, MNRAS,416,393B) \\
(2) Bertelli, G.; Girardi, L.; Nasi, E.; Marigo, P.; Castelli, F. (2007,ASPC,.374,41B) \\
(3) Kroupa, Pavel (2001,MNRAS,322,231K) \\
(4) The DES colaboration et al (2015, arXiv15030254T) \\
(5) Santiago et al (2015, arXiv150105500S) \\
(6) Bertin, E.; Arnouts, S. (2010,ASCL,soft10064B) \\
    
\section{Assinaturas}

Basílio Santiago:\rule{5cm}{.1mm}  Anna Queiroz:\rule{5cm}{.1mm} 

\end{document}
